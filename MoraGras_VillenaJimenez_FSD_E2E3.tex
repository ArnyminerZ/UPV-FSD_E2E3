\documentclass{article}

% --- Dependencies ---

\usepackage{../../../../Plantilla}

\usepackage{tikz} % Para dibujar
\usepackage{xcolor,colortbl}

\newcommand{\Title}{Entregables 2 y 3}
\newcommand{\Subject}{FSD}
\newcommand{\LogoETSIT}{../../../../ETSIT-Logo.png}
\renewcommand{\Author}{Carlos Villena, Arnau Mora}

\newcommand{\trojo}{\cellcolor{red!75}}
\newcommand{\tvrde}{\cellcolor{green!75}}
\newcommand{\tambr}{\cellcolor{yellow!75}}

\begin{document}

\section*{Entregables 2 y 3}
\subsection*{Introducción}
El objetivo de la actividad es crear un grupo de dos semáforos (uno de peatones y otro de vehículos)
que funcionen autónomamente, y sigan este ciclo de funcionamiento:
\begin{enumerate}
    \item $30$ segundos peatones
\end{enumerate}
\begin{tabular}{c ccccccccccccccccccccccccccccccccccccccccccccccccccccccc}
    Vehículos &
    \trojo&\trojo&\trojo&\trojo&\trojo& \trojo&\trojo&\trojo&\trojo&\trojo&
    \trojo&\trojo&\trojo&\trojo&\trojo& \trojo&\trojo&\trojo&\trojo&\trojo&
    \trojo&\trojo&\trojo&\trojo&\trojo& \trojo&\trojo&\trojo&\trojo&\trojo&
    \tambr&\tambr&\tambr&\tambr&\tambr& \trojo&\trojo&\trojo&\trojo&\trojo&
    \trojo&\trojo&\trojo&\trojo&\trojo& \trojo&\trojo&\trojo&\trojo&\trojo&
    \trojo&\trojo&\trojo&\trojo&\trojo \\
    Peatones &
    \trojo&\trojo&\trojo&\trojo&\trojo& \trojo&\trojo&\trojo&\trojo&\trojo&
    \trojo&\trojo&\trojo&\trojo&\trojo& \trojo&\trojo&\trojo&\trojo&\trojo&
    \trojo&\trojo&\trojo&\trojo&\trojo& \trojo&\trojo&\trojo&\trojo&\trojo&
    \trojo&\trojo&\trojo&\trojo&\trojo& \trojo&\trojo&\trojo&\trojo&\trojo&
    \trojo&\trojo&\trojo&\trojo&\trojo& \trojo&\trojo&\trojo&\trojo&\trojo&
    \trojo&\trojo&\trojo&\trojo&\trojo
\end{tabular}

\end{document}
